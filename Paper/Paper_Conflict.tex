\documentclass[11pt]{article}
\usepackage{fullpage}
\usepackage{amsmath,graphicx,color,listings,arydshln,bm,natbib,setspace}
\usepackage{booktabs}
\usepackage{siunitx}
\usepackage{dcolumn}
\usepackage{rotating}
\usepackage{bbm}
\usepackage{float}
\usepackage{courier}
\usepackage[pdftitle={},pdfauthor={},colorlinks=true,linkcolor=blue,citecolor=blue,filecolor=black,urlcolor=blue]{hyperref}

\usepackage{etoolbox}
\robustify\bfseries

\sisetup{%
	table-align-uncertainty=true,
	separate-uncertainty=true,
	detect-weight=true,
	detect-inline-weight=math
}

\title{Commodity Price Shocks and the Seasonality of Civil Unrest}
\author{David Ubilava\thanks{School of Economics, University of Sydney. Correspondence: david.ubilava@sydney.edu.au}}
%\date{This Draft: \today \\ First Draft: April 8, 2020}

\doublespacing

\begin{document}

\begin{titlepage}
	
	\maketitle
	\thispagestyle{empty}
	
	\begin{abstract}
		
		\noindent Commodity prices affect income, and can influence civil unrest in conflict-prone low-income countries. To the extent that agriculture facilitates employment and food security in these countries, conflict and violence often are linked to this sector. A farmer may turn into a fighter if income from agricultural production drops; alternatively, a fighter is more likely to extort a farmer when the value of supplies increases. The seasonality of agricultural production may lend itself to the intra-annual changes in the opportunity cost of insurrection, and the opportunities for fighters to fund themselves by appropriating farmers' supplies. In this study, I investigate the degree to which monthly cereal prices have impacted civil unrest across the African continent during the 1997--2019 period. I find that battles (involving governmental forces, rebels, or affiliated parties) are more likely during the pre-harvest season, possibly as a strategic move to appropriate expected returns; violence by and against civilians is more likely during the post-harvest season, possibly as a consequence or a repercussion of rapacity. 
		
		\medskip
		
		\noindent\textbf{Keywords}: Africa; Civil Unrest; Commodity Prices; Conflict; Riots; Violence.
		
	\end{abstract}

\end{titlepage}

\section{Introduction}

Income shocks alter people's behavior---including bad behavior---in many ways. In fragile states with weak institutions, where some variant of civil unrest is more of a norm than an exception, a sudden change in income may bring about a whole range of unlawful activities, such as rioting and looting, and, in some instances, more severe forms of armed violence. In predominantly agrarian economies, a bad harvest or a sudden drop in crop prices may constitute a great deal of a negative income shock. To that end, empirical evidence presents strong linkages between crop yields and conflict \citep{buhaug2015,koren2017,koren2018}, as well as commodity price shocks and conflict \citep{maystadt2014,smith2014,berman2015,fjelde2015,raleigh2015,crost2020,mcguirk2020}. 

That agriculture plays a crucial role in the income--conflict nexus is hardly debatable. Studies have examined a range of theories that explain the linkage between exogenous price/income shocks and conflict incidence. Many of these theories are based on the notion of a trade--off between `farming' and `fighting,' wherein income from the former is an opportunity cost of the latter. That is, opportunity cost of fighting is an increasing function of income---a negative income shock thus leading to more violence \citep{collier1998,bazzi2014}. By the same token, a drop in household income reduces the value of the `prize' to be appropriated, thus leading to less violence \citep{berman2015}. Therefore, the (net) effect of the price/income shocks on conflict is ambiguous. Indeed, previous studies have found (and justified) a positive relationship \citep[e.g.,][]{crost2020,mcguirk2020}, a negative relationship \citep[e.g.,][]{berman2015,fjelde2015}, or a no meaningful relationship \citep[e.g.,][]{bazzi2014} between commodity price shocks and conflict. 

In examining the relationship between income shocks and armed violence, the earlier studies have relied on annual commodity prices and country--level conflict measures \citep[e.g.,][]{miguel2004,bruckner2010,bazzi2014}. The signal can be lost in such temporally and spatially aggregated data, however. In a low income country, for example, a couple of months worth of negative income shock may drastically deteriorate a household's budget and, thus, increase a person's propensity to engage in unlawful activities \citep[e.g.,][]{bellemare2015}. Within a country, incentives to engage in violence of some sort can vary, for example, with the proximity to a city or a resource--abundant site \citep[e.g.,][]{berman2017,koren2017}. The more recent studies have addressed these issues by examining high-resolution grid-cell data, typically at annual frequency \citep[e.g.,][]{fjelde2015,berman2015,berman2017,koren2018}, or by examining country-level (or, in some instances, administrative-level) data at monthly frequency \citep[e.g.,][]{maystadt2014,smith2014,bellemare2015}. The present study combines these two approaches, and applies high-resolution grid-cell commodity price and civil unrest data recorded at monthly frequency. In so doing, this study is able to unveil the potential seasonal pattern of conflict and violence in economies where agriculture is the key source of income.

A case in point is the potential seasonality in the opportunity cost of farmers to engage in armed violence on the one hand, and opportunities for fighters to fund themselves by extorting farmers on the other hand. Both these effects are likely to amplify around the harvest season, and gradually decrease as the post-harvest season progresses into the planting season. 

The central question of this study is how, and to what extent, do price shocks to locally produced major cereal crops impact the probability of civil unrest in fragile countries. The geographic focus of the analysis is Africa. 

Although most African countries primarily export metals and cash crops (e.g., coffee and cocoa), the largest sector, in terms of employment, is agriculture. To that end, recent studies have examined the impact of agricultural shocks on conflict incidence \citep{berman2015,fjelde2015,mcguirk2020}. In particular, \cite{berman2015} and \cite{fjelde2015} both find the negative relationship between agricultural price shocks and conflict incidence in Africa. 

The present study echoes a subset of foregoing literature in that it also finds negative relationship between cereal grain price shocks and different forms of civil unrest. As its unique contribution, this study finds that the impact of price shocks is most pronounced on armed clash and violence during the harvest season, and on protests and riots after the harvest season. These findings are pertinent to Sub-Saharan Africa, and are robust to a range of different specifications. In what follows, I describe the data used in the analysis, followed by empirical results, and discussion of main findings, as well as mechanisms that help propagate the price--conflict relationship.

%Here, distinction must be made between rural and urban households \citep{mcguirk2020}. In rural households, the effect is likely to be most evident in labor--intensive sectors, such as food and agricultural production, wherein a negative price shock will reduce household income and, thus, the opportunity cost of armed violence. In urban household, the effect is likely to go the other direction, as the reduction of real income is typically associated with a price surge on food items. 

%Other possible mechanisms that link economic shocks with conflict and violence are, perhaps, better manifested at lower (e.g., annual) frequency. For example, the so-called \textit{rent--seeking} theory suggests that a positive price shock to the capital--intensive energy and mining sectors will increase the value of the `prize' to be seized through the capture of the state or the territory \citep{berman2017}. This can lead to the onset of a civil conflict---armed violence being both the means as well as the consequence of it. To begin, the conflict (say, with rebels) occupies much of the institutional bandwidth of the state trying to retain the `prize,' as a result, less attention and resources are directed toward maintaining law and order. This can be exacerbated by already sub-par capacity and institutions in resource--abundant states that tend rely on resource rents, making them more fragile \citep{berman2017}. Moreover, civilian co-optation is less likely in fragile states during the times of conflict, which further amplifies the frequency of violence against civilians \citep{koren2017}. 

%Increase in prices of resources increases income of a few, relative to the rest of the population, which creates incentives for the latter to target the former though robbery and abduction. 

\section{Data}

I apply publicly available data on civil unrest, prices, and crop harvest details from multiple online sources (described below). The data span the January 1997 -- December 2019 period, and cover 53 countries (10266 grid-cells) across Africa.
%I sourced the commodity price indices from the International Monetary Fund (IMF) website.\footnote{available at: \href{https://www.imf.org/en/Research/commodity-prices}{https://www.imf.org/en/Research/commodity-prices}} I sourced the conflict data from the Armed Conflict Location \& Event Data (ACLED) website.\footnote{available at: \href{https://acleddata.com/}{https://acleddata.com/}} I describe these two sets of data below. In addition, I collected the data on cities from the World Cities Database;\footnote{available at: \href{https://simplemaps.com/data/world-cities}{https://simplemaps.com/data/world-cities}} 
%the land suitability data from the Socioeconomic Data and Applications Center;\footnote{available at: \href{https://sedac.ciesin.columbia.edu/data/collection/lulc/sets/browse}{https://sedac.ciesin.columbia.edu/data/collection/lulc/sets/browse}} 
%and the population data from the European Commission Global Human Settlement database \citep{ghs-pop_2019}.\footnote{available at: \href{https://ghsl.jrc.ec.europa.eu/datasets.php}{https://ghsl.jrc.ec.europa.eu/datasets.php}}

\subsection{Civil Unrest}
I sourced the civil unrest data from the Armed Conflict Location \& Event Data (ACLED) website.\footnote{available at: \href{https://acleddata.com/}{https://acleddata.com/}} The current version of the ACLED database groups events into six distinct categories. Of these, in the present analysis I use events from categories referred to as `battles,' `violence against civilians,' `protests,' and `riots.' That is, I discard `strategic developments,' and `explosions/remote violence,' as these typically involve longer-term and larger scale conflict incidents between government and rebels, and are less likely to be triggered by monthly price shocks. To borrow \cite{mcguirk2020}'s terminology, the focus of this study is the `output conflict' rather than the `factor conflict.' 

I construct three measures of civil unrest: (i) \textbf{conflict}, which is `armed clash' sub-event from the `battles' category of conflicts that necessarily include interaction of some form between state forces, rebel groups, political militias, and identity militias; (ii) \textbf{violence}, which involves `attack' and `abduction/forced disappearance' sub-events from the `violence against civilians' category, as well as select events from `protests' and `riots' categories that necessarily include civilians as victims of interaction/altercation; and (iii) \textbf{protests}, which include events from `protests' and `riots' categories that do not include civilians as victims of interaction/altercation. From here forward, I refer the combination (sum) of these three measures as \textbf{civil unrest}. Finally, because not all reported incidents are measured with precision, in this study, I drop those events with the geo-precision code 3, which assigns a conflict to a provincial capital. I maintain all time-precision levels, as the least accurate code in the dataset still gives the correct month. 

%Figure~\ref{fig:conflict_bar} presents the cumulative recorded incidents across the countries during the study period. Figures~\ref{fig:conflict_panel}~and~\ref{fig:conflict_maps} illustrate the within--country conflict dynamics and the geographic distribution of conflicts across Africa. In Figure~\ref{fig:conflict_panel}, the shaded regions of panels represent years in which some form of change in control of territory took place---which is often and typically indicative of a civil war or a large--scale military event in the country.
%
%\begin{figure}[!htbp]
%	\centering
%	\includegraphics[width=\textwidth]{Figures/conflict_bar}
%	\caption{Civil Unrest Across African Countries During the 1997--2019 Period}
%	\label{fig:conflict_bar}
%\end{figure}
%
%\begin{figure}[!htbp]
%	\centering
%	\includegraphics[width=\textwidth]{Figures/unrest_panel_long}
%	\caption{Civil Unrest Over Time Across African Countries}
%	\label{fig:conflict_panel}
%\end{figure}
%
%\begin{figure}[!htbp]
%	\includegraphics[scale=.33]{Figures/conflict_05deg_map}
%	\includegraphics[scale=.33]{Figures/violence_05deg_map}
%	\includegraphics[scale=.33]{Figures/protests_05deg_map}
%	\caption{Intensity of Civil Unrest Across Africa}
%	\label{fig:conflict_maps}
%\end{figure}





\subsection{Production}

I obtained the data on crop production and their growing seasons from \cite{sacks2010}. For each crop, in each grid-cell, I obtain the fraction of the harvested area, as well as the planting and harvesting months, which includes the months that fall within the start and end periods. The harvest season for a given crop then is calculated as harvest months weighted by the fraction of the harvested area (likewise, the planting season is calculated as planing months weighted by the fraction of the harvested area). These crop-specific harvest (and planting) seasons are aggregated across the five major cereal crops, thus forming the crop harvesting season for a given grid-cell.

%\begin{figure}
%	\centering
%	\includegraphics[scale=.65]{Figures/crops}
%	\caption{Geographic Distribution of Major Cereal Crops Harvested Across Africa}
%\end{figure}
%
%\begin{figure}
%	\centering
%	\includegraphics[scale=1]{Figures/harvest_05deg_panel_long}
%	\caption{Spatio--Temporal Distribution of the Post-Harvest Season}
%\end{figure}



\subsection{Prices}

%\begin{figure}
%	\includegraphics[width=\textwidth]{Figures/price_series_small}
%	\caption{Price Series Over Time Across Grid Cells}
%\end{figure}

%\begin{figure}
%	\includegraphics[scale=1]{../R/Figures/prices_panel_rw}
%	\caption{The Commodity Terms of Trade Indices}
%\end{figure}





\subsection{Descriptive Statistics} 

\section{Model}

In what follows, I denote cell with subscript $i$, and month-year period with subscript $t$. The baseline specification is given by:
\begin{equation}
\text{conflict}_{it} = \beta \text{shock}_{it} + \mu_i + \lambda_{t} + \varepsilon_{it}
\end{equation}
where $\text{conflict}_{it}$ takes on value of one if at least one incident was reported within a cell $i$ in period $t$, and zero otherwise; $\text{shock}_{it}=\Delta \ln P_{it}\times A_i$ is a corresponding price shock, where $\Delta$ is the first--difference operator, and $P_{it} = \sum_{j=1}^{J} \omega_{ij} P_{jt}$ where $\omega_{ij}$ is a weight assigned to a commodity $j$, and $P_{jt}$ is the the IMF price index for commodity $j$ in period $t$; $\mu_i$ and $\lambda_{t}$ are cell and period fixed effects; $\varepsilon_{it}$ is an error term. 

\section{Results}


% Table created by stargazer v.5.2.2 by Marek Hlavac, Harvard University. E-mail: hlavac at fas.harvard.edu
% Date and time: Wed, Jul 01, 2020 - 1:55:39 PM
% Requires LaTeX packages: dcolumn 
\begin{table}[!htbp] \centering 
	\caption{Baseline} 
	\label{} 
	\begin{tabular*}{1.0\textwidth}{@{\extracolsep{\linewidth minus\linewidth}}lD{.}{.}{-3} D{.}{.}{-3} D{.}{.}{-3} D{.}{.}{-3} } 
		\toprule
		& \multicolumn{4}{c}{\textit{Dependent variable:}} \\ 
		\cline{2-5} 
		& \multicolumn{1}{c}{unrest\_dum} & \multicolumn{1}{c}{conflict\_dum} & \multicolumn{1}{c}{violence\_dum} & \multicolumn{1}{c}{protests\_dum} \\ 
		& \multicolumn{1}{c}{(1)} & \multicolumn{1}{c}{(2)} & \multicolumn{1}{c}{(3)} & \multicolumn{1}{c}{(4)}\\ 
		\midrule 
		price:agri & 0.036^{***} & 0.008^{*} & 0.019^{*} & 0.040^{***} \\ 
		& (0.013) & (0.004) & (0.011) & (0.012) \\ 
		& & & & \\ 
		\midrule
%		FALSE &  &  &  &  \\ 
		Observations & \multicolumn{1}{c}{2,823,204} & \multicolumn{1}{c}{2,823,204} & \multicolumn{1}{c}{2,823,204} & \multicolumn{1}{c}{2,823,204} \\ 
%		R$^{2}$ & \multicolumn{1}{c}{0.440} & \multicolumn{1}{c}{0.280} & \multicolumn{1}{c}{0.322} & \multicolumn{1}{c}{0.425} \\ 
		Adjusted R$^{2}$ & \multicolumn{1}{c}{0.435} & \multicolumn{1}{c}{0.274} & \multicolumn{1}{c}{0.316} & \multicolumn{1}{c}{0.420} \\ 
%		Residual Std. Error (df = 2798399) & \multicolumn{1}{c}{83.360} & \multicolumn{1}{c}{49.150} & \multicolumn{1}{c}{66.639} & \multicolumn{1}{c}{67.471} \\ 
		\bottomrule
		\textit{Note:}  & \multicolumn{4}{r}{$^{*}$p$<$0.1; $^{**}$p$<$0.05; $^{***}$p$<$0.01} \\ 
	\end{tabular*} 
\end{table} 


% Table created by stargazer v.5.2.2 by Marek Hlavac, Harvard University. E-mail: hlavac at fas.harvard.edu
% Date and time: Wed, Jul 01, 2020 - 1:58:53 PM
% Requires LaTeX packages: dcolumn 
\begin{table}[!htbp] \centering 
	\caption{Seasonal} 
	\label{} 
	\begin{tabular*}{1.0\textwidth}{@{\extracolsep{\linewidth minus\linewidth}}lD{.}{.}{-3} D{.}{.}{-3} D{.}{.}{-3} D{.}{.}{-3} } 
		\toprule
		& \multicolumn{4}{c}{\textit{Dependent variable:}} \\ 
		\cline{2-5} 
		& \multicolumn{1}{c}{unrest\_dum} & \multicolumn{1}{c}{conflict\_dum} & \multicolumn{1}{c}{violence\_dum} & \multicolumn{1}{c}{protests\_dum} \\ 
		& \multicolumn{1}{c}{(1)} & \multicolumn{1}{c}{(2)} & \multicolumn{1}{c}{(3)} & \multicolumn{1}{c}{(4)}\\ 
		\midrule
		postharv & -0.002 & -0.000 & -0.001 & -0.001 \\ 
		& (0.003) & (0.002) & (0.002) & (0.002) \\ 
		& & & & \\ 
		price:agri & 0.035^{**} & 0.006 & 0.017 & 0.039^{***} \\ 
		& (0.015) & (0.006) & (0.011) & (0.014) \\ 
		& & & & \\ 
		agri:postharv & -0.004 & -0.017 & -0.020 & 0.000 \\ 
		& (0.067) & (0.037) & (0.055) & (0.070) \\ 
		& & & & \\ 
		price:agri:postharv & 0.003 & 0.004 & 0.005 & 0.002 \\ 
		& (0.015) & (0.008) & (0.012) & (0.015) \\ 
		& & & & \\ 
		\midrule
%		FALSE &  &  &  &  \\ 
		Observations & \multicolumn{1}{c}{2,823,204} & \multicolumn{1}{c}{2,823,204} & \multicolumn{1}{c}{2,823,204} & \multicolumn{1}{c}{2,823,204} \\ 
%		R$^{2}$ & \multicolumn{1}{c}{0.440} & \multicolumn{1}{c}{0.280} & \multicolumn{1}{c}{0.322} & \multicolumn{1}{c}{0.425} \\ 
		Adjusted R$^{2}$ & \multicolumn{1}{c}{0.435} & \multicolumn{1}{c}{0.274} & \multicolumn{1}{c}{0.316} & \multicolumn{1}{c}{0.420} \\ 
%		Residual Std. Error (df = 2798396) & \multicolumn{1}{c}{83.358} & \multicolumn{1}{c}{49.149} & \multicolumn{1}{c}{66.638} & \multicolumn{1}{c}{67.468} \\ 
		\bottomrule
		\textit{Note:}  & \multicolumn{4}{r}{$^{*}$p$<$0.1; $^{**}$p$<$0.05; $^{***}$p$<$0.01} \\ 
	\end{tabular*} 
\end{table} 


% Table created by stargazer v.5.2.2 by Marek Hlavac, Harvard University. E-mail: hlavac at fas.harvard.edu
% Date and time: Wed, Jul 01, 2020 - 1:59:16 PM
% Requires LaTeX packages: dcolumn 
\begin{table}[!htbp] \centering 
	\caption{Combined} 
	\label{} 
		\begin{tabular*}{1.0\textwidth}{@{\extracolsep{\linewidth minus\linewidth}}lD{.}{.}{-3} D{.}{.}{-3} D{.}{.}{-3} D{.}{.}{-3} } 
		\toprule 
		& \multicolumn{4}{c}{\textit{Dependent variable:}} \\ 
		\cline{2-5} 
		& \multicolumn{1}{c}{unrest\_dum} & \multicolumn{1}{c}{conflict\_dum} & \multicolumn{1}{c}{violence\_dum} & \multicolumn{1}{c}{protests\_dum} \\ 
		& \multicolumn{1}{c}{(1)} & \multicolumn{1}{c}{(2)} & \multicolumn{1}{c}{(3)} & \multicolumn{1}{c}{(4)}\\ 
		\midrule 
		change & 0.114^{***} & 0.129^{***} & 0.079^{***} & 0.006 \\ 
		& (0.022) & (0.017) & (0.016) & (0.020) \\ 
		& & & & \\ 
		postharv & -0.002 & -0.000 & -0.001 & -0.001 \\ 
		& (0.003) & (0.002) & (0.002) & (0.002) \\ 
		& & & & \\ 
		price:agri & 0.031^{**} & 0.004 & 0.015 & 0.036^{**} \\ 
		& (0.015) & (0.005) & (0.010) & (0.014) \\ 
		& & & & \\ 
		agri:change & -0.820^{***} & -0.268 & -0.184 & -0.611^{*} \\ 
		& (0.316) & (0.176) & (0.195) & (0.316) \\ 
		& & & & \\ 
		agri:postharv & -0.011 & -0.022 & -0.026 & -0.005 \\ 
		& (0.066) & (0.036) & (0.055) & (0.069) \\ 
		& & & & \\ 
		price:agri:change & 0.227^{***} & 0.084^{**} & 0.086^{**} & 0.177^{**} \\ 
		& (0.072) & (0.041) & (0.041) & (0.073) \\ 
		& & & & \\ 
		price:agri:postharv & 0.005 & 0.005 & 0.007 & 0.003 \\ 
		& (0.015) & (0.008) & (0.012) & (0.015) \\ 
		& & & & \\ 
		price:agri:change:postharv & -0.044^{***} & -0.027^{**} & -0.044^{**} & -0.034 \\ 
		& (0.017) & (0.014) & (0.019) & (0.022) \\ 
		& & & & \\ 
		\midrule 
%		FALSE &  &  &  &  \\ 
		Observations & \multicolumn{1}{c}{2,823,204} & \multicolumn{1}{c}{2,823,204} & \multicolumn{1}{c}{2,823,204} & \multicolumn{1}{c}{2,823,204} \\ 
%		R$^{2}$ & \multicolumn{1}{c}{0.442} & \multicolumn{1}{c}{0.285} & \multicolumn{1}{c}{0.324} & \multicolumn{1}{c}{0.426} \\ 
		Adjusted R$^{2}$ & \multicolumn{1}{c}{0.437} & \multicolumn{1}{c}{0.278} & \multicolumn{1}{c}{0.318} & \multicolumn{1}{c}{0.421} \\ 
%		Residual Std. Error (df = 2798392) & \multicolumn{1}{c}{83.226} & \multicolumn{1}{c}{48.987} & \multicolumn{1}{c}{66.544} & \multicolumn{1}{c}{67.423} \\ 
		\bottomrule 
		\textit{Note:}  & \multicolumn{4}{r}{$^{*}$p$<$0.1; $^{**}$p$<$0.05; $^{***}$p$<$0.01} \\ 
	\end{tabular*} 
\end{table} 


\newpage
\bibliographystyle{chicago}
\bibliography{mybib}

\appendix

\section{Tables}

\begin{table}[H]
	\caption{A sample of ACLED records reported as `violence against civilians'}
	\small
	\begin{tabular*}{1.0\textwidth}{@{\extracolsep{\linewidth minus\linewidth}}rlp{4.0in}}
		\toprule
		Date	&	Country 	& Note	\\[2pt]
		\midrule
		22/06/2014	&	Sudan	&	Government-backed militiamen abducted 5 residents of Abu Rumayl village, Kolbus locality, and pillaged 6 villages in Kolbus area	\\
		20/09/2006	&	Burundi	&	In Giko, FNL faction attacked civilians, police intervened and killed one member. They also disarmed about 60 of Sindayigaya rebels, who came peacefully.	\\
		12/01/2016	&	Uganda	&	A journalist was severely beaten by police while covering a procession by the Unemployed Youths of Uganda (UAU), to the police headquarters in Naguru.	\\
		10/11/1998	&	Nigeria	&	Four people including the leader of the OPC were killed by police forces when they were demonstrating without authorization."	\\
		4/03/2017	&	Mali	&	2 civilians were killed and 3 abducted by suspected Islamist militants in Funtonde, 8km from Boni.	\\
		27/09/2012	&	Kenya	&	MRC attacks village and kills elder	\\
		24/10/2015	&	Sudan	&	Gunmen abduct 3 residents of Dankoj camp while they worked on their farm in Wadi Bardi in Saraf Omra.	\\
		8/07/2015	&	Libya	&	A man who supplied local troops with rations was kidnapped on 8 July by unidentified men. He was seized near Ajdabiyas Security Directorate.	\\
		28/02/2018	&	Sierra Leone	&	SLPP members were seriously beaten and wounded by APC thugs. At the time of writing this release, three SLPP members are currently hospitalised in critical conditions suffering from stabbings and broken limbs.	\\
		17/02/2016	&	South Sudan	&	Soldiers kill two unarmed young men on February 17 in the village of Natabo, west of Wau.	\\
		\bottomrule
	\end{tabular*}
	\begin{flushleft}
		\footnotesize
		\textit{Note}: these events were randomly sampled from a subset of countries and the time-frame used in the analysis.
	\end{flushleft}
\end{table}

\end{document}

